\documentclass{article}
\usepackage{graphicx, babel} % Required for inserting images


\title{Algorytmy hashujące}
\author{Krzysztof Sławek 72902}
\date{Kwiecien 2024}

\begin{document}

\maketitle

\section{Wstep}
Problem, który postaram sie przybliżyc to temat związany z algorytmami hashowania.
Najpierw zacznę od przedstawienia czym jest hashowanie.

Hashowanie to generowanie ciagu znakow o stałej długości na podstawie danych wejściowych o dowolnej wielkości. Wyniekiem takieog hashowania jest losowy ciag znaków, który w przypadku tych samych danych wejsciowych generuje identyczny ciąg znaków zwany hashem. Najmniejsza zmiana w danych wejściowych zmienia hash. 
Hashowanie jest procesem nieodwracalnym co w praktyce znaczy, że nie możemy odtworzyć z hasha danych, które były użyte na wejściu.

\section{Przykąłdy zastosowań algorytmów hashujących}


\section{Przykłady algorytmów hashujących}




\end{document}
