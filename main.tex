\documentclass{article}
\usepackage{graphicx, babel} % Required for inserting images


\title{Algorytmy hashujące}
\author{Krzysztof Sławek 72902}
\date{Kwiecien 2024}

\begin{document}

\maketitle

\section{Wstep}

Problem, który postaram sie przybliżyc to temat związany z algorytmami hashowania.
Najpierw zacznę od przedstawienia czym jest hashowanie.

Hashowanie to generowanie ciagu znakow o stałej długości na podstawie danych wejściowych o dowolnej wielkości. Wyniekiem takieog hashowania jest losowy ciag znaków, który w przypadku tych samych danych wejsciowych generuje identyczny ciąg znaków zwany hashem. Najmniejsza zmiana w danych wejściowych zmienia hash. 
Hashowanie jest procesem nieodwracalnym co w praktyce znaczy, że nie możemy odtworzyć z hasha danych, które były użyte na wejściu.

\section{Przykłady zastosowań algorytmów hashujących}

\begin{itemize}
    \item Bezpieczne przechowywanie haseł użytkowników w bazach danych.
    \item Weryfikacja integralności danych w transmisji 
    (np. w protokołach bezpieczeństwa internetowego, takich jak TLS/SSL).
    \item Generowanie unikalnych identyfikatorów dla plików (np. kontrola wersji, kontrola integralności plików).
    \item Zabezpieczanie danych w systemach autoryzacji i uwierzytelniania.
    \item Weryfikacja autentyczności dokumentów (np. cyfrowe podpisy).
    \item Bezpieczne przechowywanie kluczy kryptograficznych.
    \item Generowanie skrótów dla danych w celu szybkiego porównywania (np. w bazach danych lub strukturach danych).
    \item Zabezpieczanie haseł w aplikacjach mobilnych i serwisach internetowych.
    \item Zapobieganie podstawianiu plików (ang. file spoofing) przez sprawdzanie ich skrótów.
    \item Utrzymywanie integralności danych w blockchainie.
\end{itemize}

\section{Przykłady algorytmów hashujących}




\end{document}
