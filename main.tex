\documentclass{article}
\usepackage{graphicx, babel} % Required for inserting images
\usepackage{hyperref}
\usepackage{geometry}
\geometry{top=1cm, bottom=2cm, left=2cm, right=2cm}

\title{Algorytmy hashujące}
\author{Krzysztof Sławek 72902}
\date{Kwiecien 2024}

\begin{document}

\maketitle

\section{Wstep}

Problem, który postaram sie przybliżyc to temat związany z algorytmami hashowania.
Najpierw zacznę od przedstawienia czym jest hashowanie.
\\
\\
Hashowanie to generowanie ciagu znakow o stałej długości na podstawie danych wejściowych o dowolnej wielkości. Wyniekiem takiego hashowania jest losowy ciag znaków, który w przypadku tych samych danych wejsciowych generuje identyczny ciąg znaków zwany hashem. Najmniejsza zmiana w danych wejściowych zmienia hash. 
Hashowanie jest procesem nieodwracalnym co w praktyce znaczy, że nie możemy odtworzyć z hasha danych, które były użyte na wejściu.

\section{Przykłady zastosowań algorytmów hashujących}

\begin{itemize}
    \item Bezpieczne przechowywanie haseł użytkowników w bazach danych.
    \item Weryfikacja integralności danych w transmisji 
    (np. w protokołach bezpieczeństwa internetowego, takich jak TLS/SSL).
    \item Generowanie unikalnych identyfikatorów dla plików (np. kontrola wersji, kontrola integralności plików).
    \item Zabezpieczanie danych w systemach autoryzacji i uwierzytelniania.
    \item Weryfikacja autentyczności dokumentów (np. cyfrowe podpisy).
    \item Bezpieczne przechowywanie kluczy kryptograficznych.
    \item Generowanie skrótów dla danych w celu szybkiego porównywania (np. w bazach danych lub strukturach danych).
    \item Zabezpieczanie haseł w aplikacjach mobilnych i serwisach internetowych.
    \item Zapobieganie podstawianiu plików (ang. file spoofing) przez sprawdzanie ich skrótów.
    \item Utrzymywanie integralności danych w blockchainie.
\end{itemize}

\newpage
\section{Przykłady algorytmów hashujących}

\begin{enumerate}
    \item MD5 (Message Digest Algorithm 5)
    \begin{itemize}
        \item[]Do czego są najczęściej wykorzystywane: MD5 był kiedyś używany do bezpiecznego przechowywania haseł, weryfikacji integralności plików i innych zastosowań, ale ze względu na swoje słabości kryptograficzne nie jest już zalecany do tych celów.
        \\
        \item[] Jak działają: MD5 operuje na wiadomościach o zmiennej długości i zwraca 128-bitowy skrót (32 znaki szesnastkowe). Algorytm ten wykonuje szereg przekształceń na blokach danych wejściowych, w wyniku czego powstaje skrót. 
        \\
        \item[] Złożoność czasowa: $O(n)$, gdzie $n$ to długość danych wejściowych.
    \end{itemize}
    
    \item SHA-256 (Secure Hash Algorithm 256-bit)
    \begin{itemize}
        \item[] Do czego są najczęściej wykorzystywane: SHA-256 jest powszechnie stosowany do weryfikacji integralności danych, generowania kluczy kryptograficznych, podpisywania cyfrowego i innych zastosowań wymagających bezpiecznej funkcji haszującej.
        \\
        \item[] Jak działają: SHA-256 operuje na blokach danych o długości 512-bitów i zwraca 256-bitowy skrót. Działa poprzez stosowanie serii przekształceń bitowych na danych wejściowych.
        \\
        \item[] Złożoność czasowa: $O(n)$, gdzie $n$ to długość danych wejściowych.
    \end{itemize}
    
    \item BCrypt
    \begin{itemize}
        \item[]Do czego są najczęściej wykorzystywane: BCrypt jest powszechnie używany do bezpiecznego przechowywania haseł w bazach danych.
        \\
        \item[] Jak działają: BCrypt jest oparty na funkcji haszującej Blowfish. Generuje unikalną sól dla każdego hasła i wykonuje szereg iteracji, co sprawia, że jest bardziej odporny na ataki brute force. 
        \\
        \item[] Złożoność czasowa: Zależna od liczby rund algorytmu BCrypt.
        Typowo $O(2^{cost})$, gdzie $cost$ to parametr określający liczbę rund.
    \end{itemize}
    
    \item SHA-3 (Secure Hash Algorithm 3)
    \begin{itemize}
        \item[] Do czego są najczęściej wykorzystywane: SHA-3 jest używany do weryfikacji integralności danych, generowania skrótów i innych zastosowań, gdzie wymagana jest bezpieczna funkcja haszująca.
        \\
        \item[] Jak działają: SHA-3 operuje na blokach danych i zwraca skrót o określonej długości. Jest oparty na innych zasadach niż starsze wersje SHA, co sprawia, że jest bardziej oporny na niektóre ataki kryptoanalizy.
        \\
        \item[] Złożoność czasowa: $O(n)$, gdzie $n$ to długość danych wejściowych.
    \end{itemize}
    
    \item Argon2
    \begin{itemize}
        \item[] Do czego są najczęściej wykorzystywane: Argon2 jest specjalnie zaprojektowany do przechowywania haseł i jest uznawany za jeden z najbezpieczniejszych algorytmów haszujących.
        \\
        \item[] Jak działają: Argon2 wykorzystuje dużą ilość pamięci i czasu obliczeń, co sprawia, że jest bardzo odporny na ataki brute force.
        \\
        \item[] Złożoność czasowa: Zależna od parametrów,
        w tym pamięci wykorzystanej przez algorytm.
        Typowo $O(m \cdot t)$, gdzie $m$ to ilość zużytej pamięci, a $t$ to liczba iteracji.
    \end{itemize}
\end{enumerate}

\newpage
\section{Odnośniki do materiałów}
\begin{itemize}
    \item \url{https://www.okta.com/identity-101/hashing-algorithms/}
    \item \url{https://en.wikipedia.org/wiki/Cryptographic_hash_function}
    \item \url{https://en.wikipedia.org/wiki/MD5}
    \item \url{https://chainkraft.com/pl/co-to-jest-hashowanie/}
    \item \url{https://en.wikipedia.org/wiki/Merkle%E2%80%93Damg%C3%A5rd_construction}
    \item \url{https://nvlpubs.nist.gov/nistpubs/FIPS/NIST.FIPS.180-4.pdf}
    \item \url{https://en.wikipedia.org/wiki/Bcrypt}
    \item \url{https://datatracker.ietf.org/doc/rfc9106/}
    \item \url{https://nvlpubs.nist.gov/nistpubs/FIPS/NIST.FIPS.202.pdf}
\end{itemize}



\end{document}
